%Metroplolis Beamer Theme: https://github.com/matze/mtheme
\documentclass[10pt]{beamer}

\setbeamercolor{background canvas}{bg=white}

\usepackage{amsmath}
\usepackage{tikz}
\usepackage{mathdots}
\usepackage{yhmath}
\usepackage{cancel}
\usepackage{color}
\usepackage{siunitx}
\usepackage{array}
\usepackage{multirow}
\usepackage{amssymb}
\usepackage{gensymb}
\usepackage{tabularx}
\usepackage{booktabs}
\usetikzlibrary{fadings}
\usetikzlibrary{patterns}
\usetikzlibrary{shadows.blur}
\usetikzlibrary{shapes}

\newcommand{\light}[1]{\textcolor{gray}{#1}}


\usetheme[progressbar=frametitle]{metropolis}
\usepackage{appendixnumberbeamer}

\usepackage{booktabs}
\usepackage[scale=2]{ccicons}

\usepackage{pgfplots}
\usepgfplotslibrary{dateplot}

\usepackage{xspace}
\newcommand{\themename}{\textbf{\textsc{metropolis}}\xspace}

\title{Genetic Circuit Design Automation}
\subtitle{Voigt et al. \textbf{Nature} 2016}
% \date{\today}
\date{}
\author{Gian Hiltbrunner}
\institute{CBB Seminar, ETH Zurich}
\titlegraphic{\vspace{4.5cm}\hfill\includegraphics[width = 5cm]{circuit.png}}

\begin{document}

\maketitle

%\begin{frame}{Table of contents}
%  \setbeamertemplate{section in toc}[sections numbered]
%  \tableofcontents%[hideallsubsections]
%\end{frame}

\begin{frame}[fragile]{Introduction}

    {\metroset{block=fill}
      \begin{block}{Situation}
        Biotechnology depends on harnessing the ability of cells to perform computational operations. 
      \end{block}
      \begin{alertblock}{Problem}
        Designing synthetic genetic circuits is time-intensive and unreliable. 
      \end{alertblock}
      \begin{exampleblock}{Proposed Resolution}
        There exist systems used in electrical engineering that describe electronic circuits in special description languages. This system could be adapted for biological applications.
      \end{exampleblock}}
      
\end{frame}

\begin{frame}{Introduction}
  \begin{itemize}[<+- | alert@+>]
    \item Circuits require precise balancing of regulator expression
    \item Many parts are combined to build a circuit, whose functions can vary according to cellular context
    \item Circuits are defined by many states, which can be hard to characterize
    \item Many regulators show toxicity at higher concentrations
    \item[$\rightarrow$] Balancing these issues is difficult, we need computational support
  \end{itemize}
\end{frame}

\begin{frame}[fragile]{Introduction}

    {\metroset{block=fill}
    
      \begin{block}{Situation}
      
        \light{Biotechnology depends on harnessing the ability of cells to perform computational operations. }
      \end{block}
      \begin{block}{Problem}
        \light{Designing synthetic genetic circuits is time-intensive and unreliable.}
      \end{block}
      \begin{exampleblock}{Proposed Resolution}
        There exist systems used in electrical engineering that describe electronic circuits in special description languages. This system could be adapted for biological applications.
      \end{exampleblock}}
      
\end{frame}

\begin{frame}{Technology}

\vspace{0.5cm}
\begin{figure}
    \centering
    \makebox[\textwidth][c]{\includegraphics[width=1\textwidth]{{gate_technology}}}%
\end{figure}

\nocite{Brophy2014PrinciplesDesign} 
\end{frame}

\begin{frame}{Methods}

\vspace{0.5cm}
\begin{figure}
    \centering
    \makebox[\textwidth][c]{\includegraphics[width=1.2\textwidth]{{overview.jpg}}}%
\end{figure}

\nocite{Nielsen2016GeneticAutomation} 
\end{frame}

\begin{frame}{Showcase}

    \begin{minipage}{.4\textwidth}
        \centering
        \begin{itemize}
            \item Insulated gates were used to create 52 circuits. 
            \item Circuit output is linked to expression of YFP.
            \item Output states were experimentally measured using flow cytometry and compared to the predictions. 
        \end{itemize}
    \end{minipage}%
    \begin{minipage}{.6\textwidth}
        \centering
        \includegraphics[width=13cm]{examples.jpg}
    \end{minipage}

\end{frame}

\begin{frame}{Challenges}
    \centering
    \makebox[\textwidth][c]{\includegraphics[width=1.2\textwidth]{{challenges}}}%
\end{frame}



\begin{frame}{Discussion}
  \begin{itemize}[<+- | alert@+>]
    \item Circuits design has been dominated by manual tinkering, Cello automates the selection and concatenation of parts considering additional constraints
    \item This allows for the design of larger more complex multi-part systems
    \item 45 out of 60 Cello designed circuits functioned as expected, the largest containing over 12 promotors, thus doubling previous records 
    \item Of 412 output states 92 \% were found to be correct.
    \item[] \begin{quote}
        \vspace{0.2cm}
        "By controlling [...] when genes are turned on to designing therapeutic agents that are programmed to sense a problem in the body and perform a therapeutic action." [R. Cloney]
    \end{quote}
  \end{itemize}
  \nocite{Cloney2016SyntheticDesign}
\end{frame}


{
    \setbeamercolor{palette primary}{fg=black, bg=yellow}
    \begin{frame}[standout]
        \vspace{3cm}
        Questions?
        \vspace{3cm}
        {
            \footnotesize
            \begin{center}\url{github.com/gianhiltbrunner/GeneticCircuitDesignAutomationTalk}\end{center}
            
            \begin{center}\ccbysa\end{center}
        } 
    \end{frame}
}

\appendix

\begin{frame}[allowframebreaks]{References}

  \bibliography{references}
  \bibliographystyle{abbrv}

\end{frame}

\begin{frame}{}
    
\end{frame}

\end{document}
